
\documentclass[a4paper,12pt]{article}

% Packages
\usepackage[utf8]{inputenc}
\usepackage{amsmath}
\usepackage{graphicx}
\usepackage{hyperref}
\usepackage{geometry}
\geometry{a4paper, margin=1in}

% Title and Author
\title{Machine Learning Experiments Report}
\author{AIML-10}
\date{\today}

\begin{document}

\maketitle
\tableofcontents
\newpage

% Abstract
\section*{Abstract}
\addcontentsline{toc}{section}{Abstract}
This report covers how we can alter the agents inside the Unity ML-Agents package, to add more human senses. Additionally, we will analyze how we can change the model's hyper-parameters and which configuration of hyper-parameters has the best performance. The objective is to find methods to improve agents' performance in dynamic environments.


% Introduction
\section{Introduction}
There are digital problems that we cannot solve using simple programming techniques. Creating agents that are able to complete a task in a dynamic environment is one of those problems. This can be solved through using Machine Learning algorithms. Machine learning is a branch of artificial intelligence that focuses on developing algorithms and models that enable systems to learn patterns and make predictions or decisions based on data, as it allows models to adapt and improve through experience in unpredictable environments. Improving the agents' performance has many utilities in real-life, such as in the field of robotics. The objective of this report is to explore ways to improve agent performance through hyper-parameter tuning and the addition of sensory models \cite{benoit2002fuzzy} that emulate human senses within the Unity ML-Agents package \cite{unityTechnologies2019mlagents} \cite{ilosvay2024unity} \cite{juliani2018unity}.

% Experiment
\section{Experiments}

\subsection{Past Work}
Throughout our work on this project we developed certain scripts that enhanced the sensory capacity of the agents. This included a more realistic vision sensor and a sound sensor. The basic ML-Agents package already contains a vision sensor, its use is unrealistic because the agent shoots vision rays from the back-side of its head and the vision cone is coupled to the rotation of the body, this means that the eyes can only point directly in front of the body, things that are impossible in the real world. With our enhancements the agents are now able to rotate their head and also hear sounds that are generated nearby. 

\subsection{Hyper-Parameter Tuning}
To test the performance, and ability to improve, of the agents we performed two types of tests. The first test would be performed on the basic version of the Agent. We generated a script that would generate random training configurations as a way to perform a random search. Through this method we can reach an approximation of the best configuration of hyper-parameters, giving us a better insight into what hyper-parameters are important for this task. 

To find the best hyper-parameter configuration we had to choose between Grid Search, Random Search and an Evolutionary Algorithm, as our main method. We chose Random Search over the others because of a few important factors. Grid Search requires us decide on predetermined values for variable and it exhaustively iterates over all of the possible combinations. Although it does provide us with a more comprehensive overview of the important of certain variables on the performance, its time complexity is high, which is further worsened by the long training times, and the values that are used are arbitrary and discrete, meaning that there could be other values, that are not included in the list, that could perform better. The evolutionary algorithm is promising, with seemingly better efficiency compared to the other methods, but the integration complexity would result in all of our efforts going into finding a way to make it work the with the Unity Editor, taking away from our other obligations. The Random Search algorithm seemed to be the worst choice, performance wise, but through research was determined to be the best option, since it can outperform Grid Search in performance \cite{JMLR:v13:bergstra12a}.


\subsection{Comparison of Basic and Enhanced Agents}
In the second experiment we put the basic and enhanced agents in an environment and, using the same hyper-parameter configuration, we tested which one would perform better at the task.

% Results
\section{Results}

\section{Conclusions}
% Resources
\section{Appendix}


% References
\addcontentsline{toc}{section}{References}
\bibliographystyle{plain}
\bibliography{references}  % Create a separate references.bib file for citations

\end{document}
